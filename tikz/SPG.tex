\documentclass[tikz,border=10pt]{standalone}
\usetikzlibrary{arrows,intersections}
\begin{document}
\begin{tikzpicture}[
    thick,
    >=stealth',
    dot/.style = {
      draw,
      fill = white,
      circle,
      inner sep = 0pt,
      minimum size = 4pt
    }
  ]
  \coordinate (O) at (0,0);
  \draw[->] (-0.3,0) -- (5,0) coordinate[label = {below:$U$}] (xmax);
  \draw[->] (0,-0.3) -- (0,5) coordinate[label = {right:$I$}] (ymax);
  \path[name path=x] (0.3,0.5) -- (6.7,4.7);
  \path[name path=y] plot[smooth] coordinates {(-0.3,2) (2,1.5) (4,2.8) (6,5)};
  \scope[name intersections = {of = x and y, name = i}]
%    \fill[gray!20] (i-1) -- (i-2 |- i-1) -- (i-2) -- cycle;
    \draw[red] (2,0) -- (2,5) node[pos=0.8, right] {$R=0$};
%    \draw[red] plot[smooth] coordinates {(-0.3,2) (2,1.5) (4,2.8) (6,5)};
%    \draw (i-1) node[dot, label = {above:$P$}] (i-1) {} -- node[left]
%      {$f(x_0)$} (i-1 |- O) node[dot, label = {below:$x_0$}] {};
%    \path (i-2) node[dot, label = {above:$Q$}] (i-2) {} -- (i-2 |- i-1)
%      node[dot] (i-12) {};
%    \draw           (i-12) -- (i-12 |- O) node[dot,
%                              label = {below:$x_0 + \varepsilon$}] {};
%    \draw[blue, <->] (i-2) -- node[right] {$f(x_0 + \varepsilon) - f(x_0)$}
%                              (i-12);
%    \draw[blue, <->] (i-1) -- node[below] {$\varepsilon$} (i-12);
%    \path       (i-1 |- O) -- node[below] {$\varepsilon$} (i-2 |- O);
%    \draw[gray]      (i-2) -- (i-2 -| xmax);
%    \draw[gray, <->] ([xshift = -0.5cm]i-2 -| xmax) -- node[fill = white]
%      {$f(x_0 + \varepsilon)$}  ([xshift = -0.5cm]xmax);
  \endscope
\end{tikzpicture}
\end{document}